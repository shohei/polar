\documentclass{article}
\title{極座標系のメモ}
\author{青木翔平}

\usepackage[usenames]{color} %used for font color
\usepackage{amssymb} %maths
\usepackage{amsmath} %maths
\usepackage{multirow}
\usepackage{graphicx}
\usepackage[utf8]{inputenc} %useful to type directly diacritic characters
\usepackage{algorithm}
\usepackage{algorithmic}

\begin{document}
\maketitle

\begin{eqnarray}
v &=& \sqrt{\dot{x}^2 + \dot{y}^2} \nonumber \\
  &=& \sqrt{ \left(\frac{d}{dt} r cos\theta \right) ^2 + \left(\frac{d}{dt}r sin\theta\right) ^2} \nonumber \\
  &=& \sqrt{ \dot{r}^2+ r^2 \dot{\theta}^2 } 
\end{eqnarray}	
	
$v=const$(線速度一定)とすると、$\dot{v} = 0$より、
	
\begin{eqnarray}
	\frac{d}{dt} \left(\sqrt{ \dot{r}^2+ r^2 \dot{\theta}^2 } \right) &=& 0 \nonumber \\
\frac{1}{2\sqrt{ \dot{r}^2+ r^2 \theta^2 }} \cdot \frac{d}{dt}\left( \dot{r}^2+r^2 \dot{\theta}^2 \right) &=& 0 \nonumber \\ 
\frac{2\dot{r}\ddot{r} + 2r\dot{r}\dot{\theta}^2 + 2r^2\dot{\theta}\ddot{\theta}}{2\sqrt{ \dot{r}^2+ r^2 \theta^2 }} &=& 0 \nonumber \\
\dot{r}\ddot{r} + r\dot{r}\dot{\theta}^2 + r^2\dot{\theta}\ddot{\theta} &=& 0  
\end{eqnarray}
	
以上を満たすように$r$と$\theta$を制御する必要がある。
	
(2)式をさらに計算する。
$\dot{\theta}=\omega$とおいて、
\begin{eqnarray}
  \dot{r}\ddot{r}+r\omega(\dot{r}\omega+r\dot{\omega})&=&0 \nonumber \\
  \dot{r}\ddot{r} &=& -r\omega\frac{d}{dt}(r\omega) 
\end{eqnarray}
$\dot{r}=x, r\omega=y$とおくと、
\begin{eqnarray}
  x\frac{dx}{dt} &=& -y\frac{dy}{dt} \nonumber \\
  xdx + ydy &=& 0	
\end{eqnarray}

(4)式は完全微分形の微分方程式である。(4)を解くと、
\begin{eqnarray}
  x^2+y^2 &=& C  \nonumber \\
  \dot{r}^2 + (r\omega)^2 &=& C \nonumber \\
  \dot{r}^2 &=& -(r\omega)^2 + C 	
\end{eqnarray}
初期条件$t=0, r=a, \dot{r}=0, \omega=\omega_0$とすると、
\begin{eqnarray*}
  C=a^2\omega_0^2	
\end{eqnarray*}
したがって、
\begin{eqnarray}
  \dot{r}^2 &=& a^2 \omega_0^2 - r^2 \omega^2 \nonumber \\
  \dot{r}^2 - a^2 \omega_0^2 &=& r^2 \omega^2 \nonumber \\
  \omega^2 &=& \frac{1}{r^2} \left\{ \dot{r}^2 - a^2 {\omega_0}^2  \right\} \nonumber \\
  \omega &=& \frac{1}{r} \sqrt{ \dot{r}^2 - a^2 {\omega_0}^2 }  
\end{eqnarray}


\begin{algorithm}                      
\caption{Bresenham's algorithm}
\label{alg1}                          
\begin{algorithmic}                  
\STATE $dx \Leftarrow x1-x0$
\STATE $dy \Leftarrow y1-y0$
\STATE $D \Leftarrow 2*dy - dx$
\STATE plot(x0,y0)
\STATE $y \Leftarrow y0$
\FOR{$x$ from $x0+1$ to $x1$}
\IF{$D > 0$}
\STATE      $y \Leftarrow y+1$
\STATE      plot(x,y)
\STATE      $D \Leftarrow D + (2*dy-2*dx)$
\ELSE
\STATE      plot(x,y)
\STATE      $D \Leftarrow D + (2*dy)$
\ENDIF
\ENDFOR
\end{algorithmic}
\end{algorithm}



後退差分表示
\begin{eqnarray*}
F(t_n) &=& \frac{d S}{d t}\Bigg|_{t=tn} \\
&=& \sqrt{\dot{r}^2+r^2 \dot{\theta}^2}\Big|_{t=tn} \\
&=& \sqrt{\left( \frac{r_n - r_{n-1}}{\Delta t} \right)^2 + \left(\frac{r_n+r_{n-1}}{2} \right)^2 \left( \frac{\theta_n - \theta_{n-1} }{\Delta t} \right)^2}	
\end{eqnarray*}

ステージの変位とスクリューの変位は比例するので
\begin{eqnarray*}
\int_{t_1}^{t_2} SNW(t)dt &=& \alpha \int_{t_1}^{t_2} F(t) dt \\
SNW\big|_{t=t_n} &=& \alpha F(t_n)  \\
&=& \frac{\alpha}{\Delta t} \sqrt{\left( r_n - r_{n-1} \right)^2 + \left(\frac{r_n+r_{n-1}}{2} \right)^2 \left( \theta_n - \theta_{n-1} \right)^2} \\
&=& \alpha^\prime \sqrt{\left( r_n - r_{n-1} \right)^2 + \left(\frac{r_n+r_{n-1}}{2} \right)^2  \left( \theta_n - \theta_{n-1} \right)^2}
\end{eqnarray*}



\end{document}